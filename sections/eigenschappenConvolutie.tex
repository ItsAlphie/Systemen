% !TeX spellcheck = en_GB
\documentclass[]{subfiles}
\begin{document}
	\section{Property's of a convolution}
	 There are six property's which are true for a convolution. These should be included on the formula sheet on the exam. However, you do need to be able to explain what they mean and how they relate to each other.
	 \subsection{Linearity}
	 \begin{equation}
	 	f\ast (ag_1+bg_2)= a(f\ast g_1)  +b(f\ast g_2)
	 \end{equation}
 	\subsection{Commutativity}
 	\begin{equation}
 		f\ast g = g \ast f
 	\end{equation}
 \subsection{associativity}
 \begin{equation}
 	f\ast (g\ast h) = (f\ast g) \ast h
 \end{equation}
\subsection{Causal functions}
\label{sec:causalFunctions}
If
\begin{equation}
	\left\{ \begin{array}{r@{\text{ = }}l}
		f(t)& 0\\
		g(t)&0
	\end{array}\right.
	\text{for } t<0
\end{equation}
we can say these functions are causal, furthermore we can state that:
\begin{align*}
	(f\ast g) (t) &= \int_{-\infty}^{\infty} f(\tau)g(t-\tau)d\tau\\
	&= \int_{0}^{t}f(\tau)g(t-\tau)d\tau
\end{align*}
\subsection{Finite interval}
if
\begin{equation}
	\left\{ \begin{array}{r@{\text{ for }}l}
		f(t)\neq 0& t\in\left[ 0,t_1\right] \\
		g(t)\neq 0&t\in \left[ 0,t_2\right] 
	\end{array}\right.
\end{equation}
then $(f\ast g)(t)\neq 0$ for the interval $\left[0,t_1+t_2 \right] $.
\subsection{Delta function}
\begin{align*}
	x(t)\ast \delta(t-t_0) &= \int_{-\infty}^{\infty}\delta(\tau-t_0)x(t-\tau)d\tau\\
	&=  x(t-t_0)
\end{align*}
This will lead to a shift of the function to the right with a distance of $t_0$. 
\end{document}
