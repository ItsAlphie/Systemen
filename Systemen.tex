% !TeX spellcheck = en_GB
\documentclass[a4paper,kul]{kulakarticle} %options: kul or kulak (default)

\usepackage[utf8]{inputenc}
\usepackage[english]{babel}
\usepackage[T1]{fontenc}
\date{Academic year 2022 -- 2023}
\address{
        Industrial Engineeringscience \\
        Maths for systems\\
        Toon Van Waterschoot}
\title{Derivations}
\author{\href{https://github.com/debber1}{Robbe Decapmaker}}
\usepackage{hyperref}
\usepackage{graphicx}
\usepackage{amsmath, amssymb, amsthm}
\usepackage{siunitx}
\usepackage{flafter}
\usepackage{pdfpages}
\usepackage{pgfplots}
\usepackage{caption}
\usepackage{subcaption}
\usepackage{datetime2}
\usepackage{subfiles}
\usepackage{tikz}
\newcommand{\Lapl}{\ensuremath{\mathcal{L}}}

\hypersetup{
	pdftitle={Derivations Maths for systems},
	pdfsubject={Love you buddy and Pieter has a huge schlong},
	pdfauthor={Robbe Decapmaker},
	pdfkeywords={}
}

\begin{document}



\maketitle
\section*{Dedicated to the ones we lost}
\section*{Build information}
This document was built on \DTMNow.
\newline
The most recent version of this document can be found on: \href{https://c7878.gitlab.io/systemen/Systemen.pdf}{https://c7878.gitlab.io/systemen/Systemen.pdf}
\section*{Introduction}

These are the derivations for maths for systems. \href{https://github.com/debber1/Systemen}{The source code for this document can be found on github.} 
https://github.com/debber1/Systemen\\
\newline
This document is \textbf{not} everything you need to know theory-wise for the exam. I am not responsible for your final exam result. It is your responsibility to properly process the other parts of this course on your own.  This document only contains mathematical derivations and the conceptual connections between them.\\

\section*{Contributors}

%The original development off this document was in Dutch, this is why some file names are still Dutch. I am far too lazy to change them.

\newpage
\subfile{sections/convolutieDefinitie}
\newpage
\subfile{sections/eigenschappenConvolutie.tex}
\newpage
\subfile{sections/convolutieStelling.tex}
\end{document}

