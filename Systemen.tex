% !TeX spellcheck = nl
\documentclass[a4paper,kul]{kulakarticle} %options: kul or kulak (default)

\usepackage[utf8]{inputenc}
\usepackage[dutch]{babel}
\usepackage[T1]{fontenc}
\date{Academiejaar 2022 -- 2023}
\address{
        Industriële Ingenieurswetenschappen \\
        Wiskunde voor sytemen \\
        Toon Van Waterschoot}
\title{Afleidingen}
\author{\href{https://github.com/debber1}{Robbe Decapmaker}}
\usepackage{hyperref}
\usepackage{graphicx}
\usepackage{amsmath, amssymb, amsthm}
\usepackage{siunitx}
\usepackage{flafter}
\usepackage{pdfpages}
\usepackage{pgfplots}
\usepackage{caption}
\usepackage{subcaption}
\usepackage{datetime2}
\usepackage{subfiles}
\usepackage{tikz}

\hypersetup{
	pdftitle={Afleidingen Wiskunde voor systemen},
	pdfsubject={Love you buddy},
	pdfauthor={Robbe Decapmaker},
	pdfkeywords={}
}

\begin{document}



\maketitle
\section*{Dedicated to the ones we lost}
\section*{Build information}
This document was built on \DTMNow.
\newline
The most recent version of this document can be found on: \href{https://c7878.gitlab.io/systemen/Systemen.pdf}{https://c7878.gitlab.io/systemen/Systemen.pdf}
\section*{Inleiding}

De afleidingen voor wiskunde voor systemen. \href{https://github.com/debber1/Systemen}{De source code is te vinden op github.}\\
https://github.com/debber1/Systemen\\
\newline
Dit document is \textbf{niet} alles wat je moet kennen van theorie voor het examen. Ik ben niet verantwoordelijk voor jouw resultaat op jouw examen. Het is jouw verantwoordelijkheid om ook nog de andere onderdelen van deze cursus op een degelijke manier te verwerken. Het is namelijk zo dat dit document enkel wiskundige afleidingen bevat.\\

\section*{Contributors}

\newpage
\subfile{sections/convolutieDefinitie}

\end{document}

